\documentclass[11pt]{article}
\usepackage[fleqn]{amsmath}
\usepackage[margin=0.75in]{geometry}
\usepackage{xcolor}
\renewcommand{\baselinestretch}{1.5} 
\begin{document}

	
\begin{center}
	{\large\textbf{Recruiting Office Placement and Sizing:  Frommer \#1(v1.3)}}
\end{center}

This was based on a 2017 ORCA Capstone and updated for a 2024 MGMT Capstone.

Warning:  this is a draft model and has not be verified nor validated.

	\subsection*{Indices}
	$i = $ Recruiting office (RO) number, $i = 1, 2, ..., R$ \\
	$j = $ Market (CBSA) number, $j = 1, 2, ..., M$ \\

	\subsection*{Sets}
	$IR_i$ = Set of markets within the inner ring (60 miles) of RO $i$ \\
	$OR_i$ = Set of markets within the outer ring (60-120 miles) of RO $i$ \\
	$AR_i$ = Set of markets within any ring of RO $i$, equals $IR_i \bigcup OR_i$ \\
	$IR^{-1}_j$ = Set of ROs for which market $j$ is an ``inner-ring" market \\
	$OR^{-1}_j$ = Set of ROs for which market $j$ is an ``outer-ring" market \\
	$AR^{-1}_j$ = Set of ROs for which market $j$ is any type of market, equals $IR^{-1}_j \bigcup OR^{-1}_j$ \\

	\subsection*{Model Parameters}
	$d_j = $ Total potential accessions from market $j$ \\
	$c = $ Expected ``supply" (i.e. recruits accessed) per recruiter \\
	$\mu = $ Recruiting office size for each office (in number of personnel) \\
	$w_{OR} = $ Outer ring reduction weight reflecting reduced recruiter effectiveness in its ``Outer Ring" markets \\
	$a_i = $ attraction factor of recruiting office $i$, values greater than $1$ indicate the RO can more effectively attract recruits than is normal, which would be a value of $1$. \\
	$n =$ Maximum number of recruiting offices that may be opened \\
	$M =$ "Big M" larger than the max expected supply from an RO, used to ensure RO indicator variable is turned on as appropriate \\
	$\epsilon =$ very small number used to ensure RO indicator variable is turned off as appropriate, can be set to $1/M$ \\
	% Grayed out parameters are not being used in the current version of the model. \\
  	% $p$ = Total recruiter personnel available \\
	% $e_{j} = $ Efficiency of recruitment from market $j$ (average is 1, 2 is twice as hard)\\
	% $\mu_L$ = Lowest allowable recruiting office size (in number of personnel) \\
	% $\mu_H$ = Highest allowable recruiting office size (in number of personnel) \\
	
		
	\subsection*{Decision Variables}
	$x_{ij} = $ Amount of recruitment ``received" (i.e. number of accessions serviced) at market $j$ from RO $i$\\
{\setlength{\mathindent}{0cm}
\[	\delta_{i} = 
	 \left\{
		 \begin{array}{ll}
			 1 & \mbox{if RO $i$ is established}, \\
			 0 & \mbox{otherwise}
			 \end{array}
			 \right.
			 \]			 
	\textcolor{gray}{$\mu_i = $ size of recruiting office (in number of personnel)}
	
	\subsection*{Objective}
	maximize $\sum\limits_{i=1}^R \sum\limits_{j \in AR_i} x_{ij}$  \qquad Maximize Total Accessions\\
	
	\subsection*{Constraints}
	\begin{equation}  \sum\limits_{j \in AR_i} x_{ij} \le c \mu, 
		\quad i=1,2,...R \quad \end{equation} 
	\mbox{(ensure recruiting transmitted does not exceed office ``supply")} 
	\begin{equation} \sum\limits_{i \in IR_j^{-1}} a_i x_{ij} +
					 w_{OR} \sum\limits_{i \in OR_j^{-1}} a_i x_{ij} 
					\le d_j, \quad j = 1, 2, ..., M \quad 
					\mbox{(limit to accessions in a given market)} \end{equation}
	
	\begin{equation}\sum\limits_{i=1}^{R} \delta_{i} \le n \quad  \mbox{(number of ROs does not exceed maximum allowed)}
	\end{equation} 
	
	Next 2 constraints insure RO indicator variables are switched on or off depending on whether the RO transmits supply or not.	
	\begin{equation}
	\sum\limits_{j \in AR_i} x_{ij} / M <= 	\delta_i, \quad i = 1, 2, ..., R 
	\end{equation}

	\begin{equation}
	\delta_i <= \sum\limits_{j \in AR_i} x_{ij} / \epsilon, \quad i = 1, 2, ..., R
	\end{equation}
	
	
	 
	\subsection*{Variable Restrictions}
	$x_{ij} \ge 0$ \\
	$\delta_{i} $ binary \\
    \textcolor{gray}{
    $\mu_i $ integer \quad (and constraints above enforce that $\mu_i \in \,  \{0, \mu_L, \ldots , \mu_H \} $ ) }\\
	
	\subsection*{Additional Explanation}
		\begin{itemize}
	\item As formulated above, the model contains no ``demand" satisfaction constraints, that is, there is no requirement for any potential accession to be obtained.  This could easily be added with a set of constraints either uniformly or on a market case-by-case basis.
		\end{itemize}
	
	

\end{document}